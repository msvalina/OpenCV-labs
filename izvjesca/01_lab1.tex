\setcounter{figure}{0}
\section{Vježba 1: Uvod u OpenCV i podešavanje radne okoline}

\subsection{Cilj vježbe}
Upoznati se s bibliotekom OpenCV. Podesiti i proučiti alate 
za prevođenje i izvršavanje izvornog koda. Napisati jednostavan 
program za određivanje rubova na slici pomoću \emph{Canny-evog}
detektora rubova. Proširiti program funkcijom rezanja slike
i dodati mogućnosti učitavnja videa.\\

\subsection{Kratak info o OpenCV-u}

OpenCV \emph{(Open Source Computer Vision Library)}
je biblioteka C i C++ funkcija koje se često upotrebljavaju u računalnom vidu. 
Početno razvijena od strane \emph{Intel-a}, a trenutno ju 
održavaju \emph{Willow Garage} i \emph{Itseez}.
Slobodna je za upotrebu pod uvijetima BSD licence. Biblioteka se izvodi na svim većim 
platformama. Fokusira se na obradu slike u stvarnom vremenu.

\subsection{Radna okolina}

\begin{description}
  \item[OS:] Ubuntu 12.04
  \item[Biblioteka:] OpenCV 2.4.6
  \item[Prevodioc:] GCC 4.6.3 
  \item[Ostalo:] pkg-config 0.26
\end{description}

Ubuntu 12.04 je izabran zbog stabilnosti, jednostavnog podešavanja i 
dostupnosti velike količine već priprmljenih programskih paketa.
OpenCV 2.4.6 je zadnja stabilna verzija u trenutnku pisanja. Odabrana
je jer se razvoj aplikacija odrađivao na više računala s različitim 
GNU/Linux distribucijama te zbog dostupnosti najnovih funkicija.
GCC 4.6.3 je zadna verzija koja dolazi s Ubuntu 12.04 distribucijom.
pkg-config alat je korišten jer on automatski izlista potrebne
opencv biblioteke i datoteke zaglavlja prevodiocu. 
\\

\begin{lstlisting}[language=bash,caption={Pokretanje prevodioca iz
    komandne linije}]
$ g++ edge_crop_video.cpp -o edge_crop_video `pkg-config --cflags --libs opencv`
\end{lstlisting}

\newpage
\subsection{Objašnjenje programa}

\subsubsection{Kontrola programa}

Prilikom pokretanja programa iz komandne linije potrebno je 
predati programu putanju do slike (argv[1]) u protivnom se program 
nece pokrenuti. \\

\begin{lstlisting}[language=bash,caption={Pokretanje programa iz
    komandne linije}]
$ ./edge_crop_video ../images/lena_color_512.tif
\end{lstlisting}

Nakon pokretanja program se kontrolira tipkovnicom, ovisno o
pritisku tipke poziva se odgovarajuća funkcija. To je izvedeno 
putem beskonačne petlje i čekanja na pritisak tipke. 
\\

\begin{lstlisting}[language=C,caption={Kontrola programa tipkovnicom}]
while ( 1 ){
    char c = waitKey(10);
    switch( c ) {
        case 27:
            cout << "Exiting ... \n ";
            return 0;
        case 'e':
            cannyEdge( loaded_img, canny_box );
            break;
        case 'r':
            cout << "Setting callback, calling cropImage  ...\n";
            setMouseCallback( imageName, onMouse, (void*)&loaded_img );
            break;
        case 'c':
            initCamera( );
            break; 
            }
        }
\end{lstlisting}

\subsubsection{Detekcija rubova}
\subsubsection{Rezanje slike}
\subsubsection{Učitavanje videa}
% \lstinputlisting{$HOME/OpenCV-labs/lab1/edge_crop_video.cpp}

