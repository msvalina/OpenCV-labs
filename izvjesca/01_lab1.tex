\setcounter{figure}{0}
\section{Vježba 1: Uvod u OpenCV i podešavanje radne okoline}

\textbf{Cilj vježbe:}\\
Upoznati se s bibliotekom OpenCV. Napisati jednostavan program za
određivanje rubova na slici pomoću \emph{Canny-evog} detektora
rubova. Podesiti i proučiti alate za prevođenje i izvršavanje izvornog
koda.\\

\subsection{OpenCV}

OpenCV \emph{(Open Source Computer Vision Library)}
je biblioteka C i C++ funkcija koje se često upotrebljavaju u računalnom vidu. 
Početno razvijena od strane \emph{Intel-a}, a trenutno ju 
održavaju \emph{Willow Garage} i \emph{Itseez}.
Slobodna je za upotrebu pod uvijetima BSD licence. Biblioteka se izvodi na svim većim 
platformama. Fokusira se na obradu slike u stvarnom vremenu.

\subsection{Radna okolina}

\begin{description}
  \item[OS:] Ubuntu 12.04
  \item[Biblioteka:] OpenCV 2.4.6
  \item[Prevodioc:] GCC 4.6.3 
  \item[Ostalo:] pkg-config 0.26
\end{description}

Ubuntu 12.04 je izabran zbog stabilnosti, jednostavnog podesavanja i 
dostupnosti velike količine vec priprmljenih programskih paketa.
OpenCV 2.4.6 je zadnja stabilna verzija u trenutnku pisanja. Odabrana
je jer se razvoj aplikacija odrađivao na više računala s različitim 
GNU/Linux distribucijama te zbog dostupnosti najnovih funkicija.
GCC 4.6.3 je zadna verzija koja dolazi s Ubuntu 12.04 distribucijom.
pkg-config alat je korišten jer on automatski izlista potrebne
opencv biblioteke i datoteke zaglavlja prevodiocu. 
\\


\begin{lstlisting}[language=bash,caption={Pokretanje prevodioca iz terminala}]
$ g++ ime_datoteke.cpp -o program `pkg-config --cflags --libs opencv`
\end{lstlisting}

\newpage
\subsection{Izlistanje koda}

% \lstinputlisting{$HOME/OpenCV-labs/lab1/edge_crop_video.cpp}

